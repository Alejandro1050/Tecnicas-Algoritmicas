\documentclass{article}
\usepackage[utf8]{inputenc}
\usepackage[spanish]{babel}
\usepackage{amsmath}
\usepackage{listings} % Para insertar código
\usepackage{xcolor} % Para colores en el código
\usepackage{graphicx}

\lstset{
    language=Python,
    basicstyle=\ttfamily\small,
    keywordstyle=\color{blue},
    commentstyle=\color{green!50!black},
    stringstyle=\color{red},
    showstringspaces=false,
    breaklines=true,
    frame=single,
    numbers=left,
    numberstyle=\tiny\color{gray},
    captionpos=b
}

\begin{document}

\begin{figure}
    \centering
    \includegraphics[width=1\linewidth]{image.png}
    \label{fig:enter-label}
\end{figure}

\title{Algoritmo de Dijkstra: Teoría e Implementación}
\author{Alejandro Cordova}
\date{10/04/2025}

\newpage

\section{Introducción}
El \textbf{algoritmo de Dijkstra} (1959) es un método para encontrar el camino más corto desde un nodo origen a todos los demás nodos en un grafo \textbf{con pesos no negativos}. Fue creado por el científico de computación neerlandés \textbf{Edsger W. Dijkstra} y publicado en 1959.

\section{Historia}
Dijkstra desarrolló este algoritmo mientras trabajaba en el \textbf{Centro Matemático de Ámsterdam}. Originalmente, lo diseñó para demostrar las capacidades de las computadoras ARMAC, pero rápidamente se convirtió en un pilar de la teoría de grafos y las redes de routing (como Internet).


\newpage
\section{Descripción del Algoritmo}
Dado un grafo $G = (V, E)$ con:
\begin{itemize}
    \item $V$: conjunto de vértices (nodos).
    \item $E$: conjunto de aristas (con pesos $w(e) \geq 0$).
\end{itemize}

El algoritmo sigue estos pasos:
\begin{enumerate}
    \item Inicializar distancias: $d(s) = 0$ (origen) y $d(v) = \infty$ para los demás nodos.
    \item Seleccionar el nodo con la distancia mínima no procesado.
    \item Relajar las aristas: actualizar las distancias de los nodos adyacentes.
    \item Repetir hasta que todos los nodos sean procesados.
\end{enumerate}

\section{Implementación en Python}
\subsection{Estructura del Algoritmo}
El siguiente código implementa Dijkstra usando:
\begin{itemize}
    \item \texttt{heapq} para gestión eficiente de colas de prioridad
    \item Diccionarios para almacenar distancias y predecesores
\end{itemize}

\begin{lstlisting}[caption=Implementación del Algoritmo de Dijkstra]

import heapq

def dijkstra(graph, start):
    # Inicializar distancias como infinito
    distances = {node: float('infinity') for node in graph}
    distances[start] = 0
    predecessors = {node: None for node in graph}
    priority_queue = [(0, start)]
    
    while priority_queue:
        current_distance, current_node = heapq.heappop(priority_queue)
        if current_distance > distances[current_node]:
            continue
        
        for neighbor, weight in graph[current_node].items():
            distance = current_distance + weight
            if distance < distances[neighbor]:
                distances[neighbor] = distance
                predecessors[neighbor] = current_node
                heapq.heappush(priority_queue, (distance, neighbor))
    
    return distances, predecessors
\end{lstlisting}

\subsection{Función Auxiliar}
\begin{lstlisting}[caption=Reconstrucción del Camino Más Corto]
def get_shortest_path(predecessors, target):
    path = []
    current = target
    
    while current is not None:
        path.append(current)
        current = predecessors[current]
    
    return path[::-1]  # Invertir el camino
\end{lstlisting}

\section{Ejemplo Práctico}
\subsection{Grafo de Prueba}
\begin{lstlisting}[caption=Ejemplo de Grafo]
graph = {
    'A': {'B': 5, 'C': 1},
    'B': {'A': 5, 'C': 2, 'D': 1},
    'C': {'A': 1, 'B': 2, 'D': 4, 'E': 8},
    'D': {'B': 1, 'C': 4, 'E': 3, 'F': 6},
    'E': {'C': 8, 'D': 3},
    'F': {'D': 6}
}
\end{lstlisting}

\subsection{Resultados}
La ejecución desde el nodo 'A' produciría:
\begin{verbatim}
Distancias más cortas desde A:
A -> A: 0
A -> B: 3
A -> C: 1
A -> D: 4
A -> E: 7
A -> F: 10

Camino más corto de A a F: A -> C -> B -> D -> F
\end{verbatim}

\section{Análisis de Complejidad}
\begin{itemize}
    \item \textbf{Tiempo}: $O((V+E) \log V)$ con cola de prioridad
    \item \textbf{Espacio}: $O(V)$ para almacenar distancias y predecesores
\end{itemize}

\section{Aplicaciones}
\begin{itemize}
    \item Sistemas de navegación (GPS).
    \item Routing en redes de computadoras.
    \item Optimización en redes de transporte.
\end{itemize}

\section{Conclusión}
El algoritmo de Dijkstra representa un hito fundamental en la teoría de grafos y la computación moderna. A través de este trabajo, hemos demostrado que:

\begin{itemize}
    \item Su \textbf{elegante diseño greedy} lo hace óptimo para resolver problemas de caminos mínimos en grafos con pesos no negativos, alcanzando una complejidad computacional eficiente de $O((V+E) \log V)$ mediante el uso de colas de prioridad.

    \item La \textbf{versatilidad de sus aplicaciones} abarca desde sistemas de navegación GPS hasta routing en redes de telecomunicaciones, demostrando su relevancia seis décadas después de su publicación.

    \item Nuestra implementación en Python evidenció cómo la combinación de diccionarios y la estructura \texttt{heapq} permite una traducción directa del fundamento matemático a código ejecutable, manteniendo claridad conceptual.
\end{itemize}

\noindent\textbf{Reflexión final:} Pese a la aparición de alternativas como el algoritmo A* para casos específicos, el legado de Dijkstra persiste como piedra angular en la enseñanza de algoritmos y como herramienta práctica en ingeniería. Su simplicidad conceptual y eficiencia lo mantienen vigente en la era de los macrodatos y las redes complejas, confirmando que los clásicos algorítmicos, cuando están bien diseñados, trascienden las barreras temporales de la tecnología.

\end{document}